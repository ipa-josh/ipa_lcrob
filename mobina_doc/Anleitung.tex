% !TEX TS-program = pdflatex
% !TEX encoding = UTF-8 Unicode

% This is a simple template for a LaTeX document using the "article" class.
% See "book", "report", "letter" for other types of document.

\documentclass[11pt]{article} % use larger type; default would be 10pt

\usepackage[utf8]{inputenc} % set input encoding (not needed with XeLaTeX)

%%% Examples of Article customizations
% These packages are optional, depending whether you want the features they provide.
% See the LaTeX Companion or other references for full information.

%%% PAGE DIMENSIONS
\usepackage{geometry} % to change the page dimensions
\geometry{a4paper} % or letterpaper (US) or a5paper or....
% \geometry{margins=2in} % for example, change the margins to 2 inches all round
% \geometry{landscape} % set up the page for landscape
%   read geometry.pdf for detailed page layout information

\usepackage{graphicx} % support the \includegraphics command and options

% \usepackage[parfill]{parskip} % Activate to begin paragraphs with an empty line rather than an indent

%%% PACKAGES
\usepackage{booktabs} % for much better looking tables
\usepackage{array} % for better arrays (eg matrices) in maths
\usepackage{paralist} % very flexible & customisable lists (eg. enumerate/itemize, etc.)
\usepackage{verbatim} % adds environment for commenting out blocks of text & for better verbatim
\usepackage{subfig} % make it possible to include more than one captioned figure/table in a single float
% These packages are all incorporated in the memoir class to one degree or another...

\usepackage{tikz} % for much better looking tables
\usepackage{framed} % for much better looking tables

\usepackage{enumerate} 
\usepackage{enumitem} 
\usepackage{lmodern} 
\usepackage{wasysym} 

\usepackage{framed,xcolor}
\colorlet{shadecolor}{red!25}

%%% HEADERS & FOOTERS
\usepackage{fancyhdr} % This should be set AFTER setting up the page geometry
\pagestyle{fancy} % options: empty , plain , fancy
\renewcommand{\headrulewidth}{0pt} % customise the layout...
\lhead{}\chead{}\rhead{}
\lfoot{}\cfoot{\thepage}\rfoot{}

%%% SECTION TITLE APPEARANCE
\usepackage{sectsty}
\allsectionsfont{\sffamily\mdseries\upshape} % (See the fntguide.pdf for font help)
% (This matches ConTeXt defaults)

%%% ToC (table of contents) APPEARANCE
\usepackage[nottoc,notlof,notlot]{tocbibind} % Put the bibliography in the ToC
\usepackage[titles,subfigure]{tocloft} % Alter the style of the Table of Contents
\renewcommand{\cftsecfont}{\rmfamily\mdseries\upshape}
\renewcommand{\cftsecpagefont}{\rmfamily\mdseries\upshape} % No bold!

%%% END Article customizations

%%% The "real" document content comes below...
\newcommand{\mb}{MobiNa}

\title{\mb{} - Anleitung}
\author{Joshua Hampp}
%\date{} % Activate to display a given date or no date (if empty),
         % otherwise the current date is printed 

\begin{document}
\maketitle

\section{Einschalten}

\begin{figure}[h]
\centering
\scalebox{0.7}{
\begin{tikzpicture}
    \tikzstyle{every node}=[font=\Large]

    \node[anchor=south west,inner sep=0] at (0,0) {\includegraphics[width=\textwidth]{imgs/irobot.jpg}};

    \draw[red,ultra thick,rounded corners] (2.8,4.7) rectangle (4.4,6);

\end{tikzpicture}
}
	\caption{Power-Knopf}
	\label{turn_on}
\end{figure}

Zum Einschalten den Deckel anheben und den Power-Knopf (Abb. \ref{turn_on}) drücken.
Daraufhin piepst \mb{} einmal und der Kopf fährt eventuell herunter.
Sobald das System komplett gestartet ist fährt der Kopf in die Ausgangsposition (senkrecht zum Boden) zurück und piepst ein zweites mal.

\section{Ausschalten}

Schalten Sie am Abend das Tablet und den Roboter aus.
Dazu befindet sich seitlich am Tablet ein Knopf über der Lautstärkeregelung (Abb. \ref{volume}). Diesen gedrückt halten bis die Option "Ausschalten" auf dem Display erscheint.
Für den Roboter drücken Sie einmal auf den Power-Knopf (Abb. \ref{turn_on}) ohne angeschlossenes Ladegerät.

\section{Bitte beachten Sie ...}

Bevor Sie \mb{} anschalten, stellen Sie bitte sicher:
\begin{itemize}
	\item Router (schwarzer Kasten mit Antennen) ist an und leuchtet (blau)
	\item der Bildschirm des Tablets ist an und ausreichend geladen (ca. 50\%)
	\item die Abdeckung befindet sich auf \mb{}
\end{itemize}

\begin{shaded}
Sollte das Tablet nicht ausreichend geladen sein, entfernen Sie das eingesteckte Kabel und nutzen Sie das beiliegende Ladegerät für das Tablet, um das Tablet zu laden.
\end{shaded}
\section{Einsatz}

Ist \mb{} nicht im aktiven Einsatz, muss \mb{} auf dem Podest stehen und das Ladegerät angeschlossen sein.
Für die Vorführung wird das Ladegerät entfernt und \mb{} auf die Markierungen gestellt.
Es sollte dabei auf die \textbf{genaue Ausrichtung} geachtet werden.

\section{Bedienung per Joystick}

Für die Steuerung von\mb{} muss zunächst die gewünschte Funktion ausgewählt werden:
\begin{description}
	\item[Totmanntaster (5)] erlaubt das fahren und bewegen des Kopfes.
	\item[LED (4)] erlaubt das verändern der Beleuchtung.
	\item[Script (Back)] startet Programme, wie z.~B. den Schnelltest.
\end{description}

\subsection{Totmanntaster}
Wenn der \textbf{Totmanntaster} gedrückt ist, können folgende Tasten verwendet werden:
\begin{description}
	\item[(1)] dreht \mb{}.
	\item[(2)] bewegt \mb{} vor und zurück.
	\item[(3) und (4)] bewegt den Kopf.
\end{description}

\subsection{LED}
Wenn der \textbf{LED-Taster} gedrückt ist, können folgende Tasten verwendet werden:
\begin{description}
	\item[(B) mit (3)] verändert die Helligkeit der roten Farbe.
	\item[(A) mit (3)] verändert die Helligkeit der grünen Farbe.
	\item[(X) mit (3)] verändert die Helligkeit der blauen Farbe.
\end{description}

\subsection{Programm}
Wenn der \textbf{Programm-Taster} (Back) und \textbf{Totmanntaster} gedrückt ist, können folgende Tasten verwendet werden:
\begin{description}
	\item[(A)] startet simulierten Sturz.
	\item[(B)] startet Schnelltest.
\end{description}

\section{Fehlerbehandlung}

\begin{figure}[h]
\centering
\scalebox{0.7}{
\begin{tikzpicture}
    \tikzstyle{every node}=[font=\Large]

    \node[anchor=south west,inner sep=0] at (0,0) {\includegraphics[width=\textwidth]{imgs/xperia.jpg}};

    \draw[red,ultra thick,rounded corners] (13.1,6.5) rectangle (14,9);

\end{tikzpicture}
}
	\caption{Lautstärkeregler des Tablets}
	\label{volume}
\end{figure}

Im Fehlerfall führen Sie zunächst den Schnelltest (siehe S. \ref{check}) aus.
Sollte kein Ton zu hören sein, prüfen Sie bitte die Lautstärke des Tablets (siehe Abbildung \ref{volume}).

\clearpage
\newpage
\begin{minipage}{\textwidth}

\section{Schnelltest}\label{check}

\begin{enumerate}
\item \mb{} per Joystick vor und rückwärts fahren
	\begin{itemize}[label={\Square}] 
	\item Hat Funktioniert?
	\end{itemize}

\item Kopf per Joystick bewegen
	\begin{itemize}[label={\Square}] 
	\item Hat Funktioniert?
	\end{itemize}

\item Schnelltest per Joystick starten und Checkliste in gegebener Reihenfolge abhacken
	\begin{description}
	\item[Schnelltest starten:] Totmanntaste + Scripttaste + Grüne Taste (A) gleichzeitig drücken
	\end{description}

\end{enumerate}

\begin{tabular}{| l | c |}
    \hline
    \textbf{Aktion} &\textbf{Erfolgreich} \\ \hline
    Rot leuchten &  \\ \hline
    Grün leuchten &  \\ \hline
    Blau leuchten &  \\ \hline
    Tablet steht senkrecht &  \\ \hline
    Tablet neight sich nach vorne &  \\ \hline
    Tablet steht senkrecht &  \\ \hline
    Grün leuchten &  \\ \hline
    Bildschirm (Tablet) an &  \\ \hline
    Film wird auf Tablet abgespielt &  \\
    \hline
  \end{tabular}

\end{minipage}

\clearpage
\newpage
\begin{minipage}{\textwidth}

\section{Tastenbelegung des Joysticks}

\begin{center}
\scalebox{0.8}{
\begin{tikzpicture}
    \tikzstyle{every node}=[font=\Large]

    \node[anchor=south west,inner sep=0] at (0,0) {\includegraphics[width=\textwidth]{imgs/joy_top.jpg}};

    \draw[red,ultra thick,rounded corners] (8.6,2.2) rectangle node {1} (10.6,4.2);
    \draw[red,ultra thick,rounded corners] (4.4,2.2) rectangle node {2} (6.4,4.2);
    \draw[red,ultra thick,rounded corners] (2,4.4) rectangle node {3} (4.5,6.9);
    \draw[red,ultra thick,rounded corners] (5.2,6.2) rectangle node {Back} (6.8,7);

\end{tikzpicture}
}
\end{center}

\begin{center}
\scalebox{0.8}{
\begin{tikzpicture}
    \tikzstyle{every node}=[font=\Large]

    \node[anchor=south west,inner sep=0] at (0,0) {\includegraphics[width=\textwidth]{imgs/joy_back.jpg}};

    \draw[red,ultra thick,rounded corners] (11.1,2.5) rectangle node {4} (13.4,4.5);
    \draw[red,ultra thick,rounded corners] (1.5,2.5) rectangle node {5} (3.7,4.5);

\end{tikzpicture}
}
\end{center}

\begin{center}
\begin{tabular}{| l | r |}
    \hline
    \textbf{Taste} &\textbf{Funktion} \\ \hline
    1 & Drehung \\ \hline
    2 & Beschleunigung \\ \hline
    3 & Kopf neigen/Helligkeit einstellen \\ \hline
    4 & Beleuchtungsfunktion auswählen \\ \hline
    Back & Programmwahl \\ \hline
    X & Blau \\ \hline
    Y & Kopf auswählen \\ \hline
    A& Grün/simulierter Sturz \\ \hline
    B & Rot/Schnelltest \\
    \hline
  \end{tabular}
\end{center}

\end{minipage}


\clearpage
\newpage
\section{Kontaktpersonen}

Bei Problemen melden Sie sich bitte bei:

\begin{framed}
{\parindent0pt
Joshua Hampp \\
Telefon +49 711 970-1843 \\
joshua.hampp@ipa.fraunhofer.de
}\end{framed}

oder

\begin{framed}
{\parindent0pt
Ralf Simon King \\
Telefon +49 711 970-1260 \\
ralf-simon.king@ipa.fraunhofer.de
}\end{framed}


\newpage
\section{Packliste}
\begin{itemize}
	\item \mb{}
	\item Netzteil für \mb{}
	\item Router
	\item Netzteil für Router
	\item 2 Netzwerkkabel
	\item Tablet
	\item Netzteil und Ladekabel für das Tablet
	\item 3 Anleitungen
%	\item Notfallumschlag mit SD-Karte
\end{itemize}

\end{document}
